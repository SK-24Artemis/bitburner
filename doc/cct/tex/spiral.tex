%%%%%%%%%%%%%%%%%%%%%%%%%%%%%%%%%%%%%%%%%%%%%%%%%%%%%%%%%%%%%%%%%%%%%%%%%%%

\section{Spiralize matrix}


%%%%%%%%%%%%%%%%%%%%%%%%%%%%%%%%%%%%%%%%%%%%%%%%%%%%%%%%%%%%%%%%%%%%%%%%%%%

\subsection*{Problem}

Given a 2-D matrix, print the elements of the matrix in spiral order,
going in clockwise direction.


%%%%%%%%%%%%%%%%%%%%%%%%%%%%%%%%%%%%%%%%%%%%%%%%%%%%%%%%%%%%%%%%%%%%%%%%%%%

\subsection*{Solution}

Consider an example matrix:
%%
\begin{equation}
\label{eqn:spiral:example_matrix}
\begin{bmatrix}
0  & 1  & 2  & 3  & 4  \\[4pt]
5  & 6  & 7  & 8  & 9  \\[4pt]
10 & 11 & 12 & 13 & 14 \\[4pt]
15 & 16 & 17 & 18 & 19
\end{bmatrix}.
\end{equation}
%%
Start from the entry $\Pair{0}{0}$ in row $0$ and column $0$,
i.e.~$0$, and navigate the outer ring of the matrix.  Move along row
$0$ from left to right and collect the numbers
$\TupleE{0}{1}{2}{3}{4}$.  Start from the rightmost column at the
entry $\Pair{0}{4}$ in row $0$ and column $4$, i.e.~$4$.  Move down
from top to bottom and collect the numbers as we go along, excluding
the entry at the coordinates $\Pair{0}{4}$, and we have
$\TupleC{9}{14}{19}$.  From the bottommost row, move along from right
to left and collect the numbers $\TupleD{18}{17}{16}{15}$.  We exclude
the entry $19$ because it was collected when we traversed along the
rightmost column.  From the entry $15$ in the leftmost column, going
from bottom to top collect numbers as we go along, resulting in
$\TupleB{10}{5}$.  We exclude $15$ because it was already collected
when we traversed the bottommost row.  Furthermore, we exclude $0$
because it was collected when we moved along the topmost row.  The
outer ring of \Matrix{eqn:spiral:example_matrix} consists of the
numbers
\[
\TupleE{0}{1}{2}{3}{4},\,
\TupleC{9}{14}{19},\,
\TupleD{18}{17}{16}{15},\,
\TupleB{10}{5}.
\]
Now consider the submatrix
%%
\begin{equation}
\label{eqn:spiral:example_matrix:submatrix}
\begin{bmatrix}
6  & 7  & 8  \\[4pt]
11 & 12 & 13
\end{bmatrix}
\end{equation}
%%
of \Matrix{eqn:spiral:example_matrix}.  Use the same procedure as
above, i.e.~move along the outer ring of
\Submatrix{eqn:spiral:example_matrix:submatrix}, and we obtain the
sequence of numbers
\[
\TupleC{6}{7}{8},\,
\TupleC{13}{12}{11}.
\]
The spiral order of \Matrix{eqn:spiral:example_matrix} in clockwise
direction is the sequence of numbers
\[
\TupleE{0}{1}{2}{3}{4},\,
\TupleC{9}{14}{19},\,
\TupleD{18}{17}{16}{15},\,
\TupleB{10}{5},\,
\TupleC{6}{7}{8},\,
\TupleC{13}{12}{11}.
\]

We have $2$ exceptional cases.  The first is a (sub)matrix consisting
entirely of $1$ row, called a row matrix.  The second is a (sub)matrix
consisting entirely of $1$ column, called a column matrix.  If we have
a row matrix, then traverse the row from left to right and we are
done.  For the case of a column matrix, traverse the column from top
to bottom.  Our discussion is summarized in
\Algorithm{alg:spiral:spiral_order_clockwise}.

\begin{algorithm}[!htbp]
%%%%%%%%%%%%%%%%%%%%%%%%%%%%%%%%%%%%%%%%%%%%%%%%%%%%%%%%%%%%%%%%%%%%%%%%%%%

\section{Spiralize matrix}


%%%%%%%%%%%%%%%%%%%%%%%%%%%%%%%%%%%%%%%%%%%%%%%%%%%%%%%%%%%%%%%%%%%%%%%%%%%

\subsection*{Problem}

Given a 2-D matrix, print the elements of the matrix in spiral order,
going in clockwise direction.


%%%%%%%%%%%%%%%%%%%%%%%%%%%%%%%%%%%%%%%%%%%%%%%%%%%%%%%%%%%%%%%%%%%%%%%%%%%

\subsection*{Solution}

Consider an example matrix:
%%
\begin{equation}
\label{eqn:spiral:example_matrix}
\begin{bmatrix}
0  & 1  & 2  & 3  & 4  \\[4pt]
5  & 6  & 7  & 8  & 9  \\[4pt]
10 & 11 & 12 & 13 & 14 \\[4pt]
15 & 16 & 17 & 18 & 19
\end{bmatrix}.
\end{equation}
%%
Start from the entry $\Pair{0}{0}$ in row $0$ and column $0$,
i.e.~$0$, and navigate the outer ring of the matrix.  Move along row
$0$ from left to right and collect the numbers
$\TupleE{0}{1}{2}{3}{4}$.  Start from the rightmost column at the
entry $\Pair{0}{4}$ in row $0$ and column $4$, i.e.~$4$.  Move down
from top to bottom and collect the numbers as we go along, excluding
the entry at the coordinates $\Pair{0}{4}$, and we have
$\TupleC{9}{14}{19}$.  From the bottommost row, move along from right
to left and collect the numbers $\TupleD{18}{17}{16}{15}$.  We exclude
the entry $19$ because it was collected when we traversed along the
rightmost column.  From the entry $15$ in the leftmost column, going
from bottom to top collect numbers as we go along, resulting in
$\TupleB{10}{5}$.  We exclude $15$ because it was already collected
when we traversed the bottommost row.  Furthermore, we exclude $0$
because it was collected when we moved along the topmost row.  The
outer ring of \Matrix{eqn:spiral:example_matrix} consists of the
numbers
\[
\TupleE{0}{1}{2}{3}{4},\,
\TupleC{9}{14}{19},\,
\TupleD{18}{17}{16}{15},\,
\TupleB{10}{5}.
\]
Now consider the submatrix
%%
\begin{equation}
\label{eqn:spiral:example_matrix:submatrix}
\begin{bmatrix}
6  & 7  & 8  \\[4pt]
11 & 12 & 13
\end{bmatrix}
\end{equation}
%%
of \Matrix{eqn:spiral:example_matrix}.  Use the same procedure as
above, i.e.~move along the outer ring of
\Submatrix{eqn:spiral:example_matrix:submatrix}, and we obtain the
sequence of numbers
\[
\TupleC{6}{7}{8},\,
\TupleC{13}{12}{11}.
\]
The spiral order of \Matrix{eqn:spiral:example_matrix} in clockwise
direction is the sequence of numbers
\[
\TupleE{0}{1}{2}{3}{4},\,
\TupleC{9}{14}{19},\,
\TupleD{18}{17}{16}{15},\,
\TupleB{10}{5},\,
\TupleC{6}{7}{8},\,
\TupleC{13}{12}{11}.
\]

We have $2$ exceptional cases.  The first is a (sub)matrix consisting
entirely of $1$ row, called a row matrix.  The second is a (sub)matrix
consisting entirely of $1$ column, called a column matrix.  If we have
a row matrix, then traverse the row from left to right and we are
done.  For the case of a column matrix, traverse the column from top
to bottom.  Our discussion is summarized in
\Algorithm{alg:spiral:spiral_order_clockwise}.

\begin{algorithm}[!htbp]
%%%%%%%%%%%%%%%%%%%%%%%%%%%%%%%%%%%%%%%%%%%%%%%%%%%%%%%%%%%%%%%%%%%%%%%%%%%

\section{Spiralize matrix}


%%%%%%%%%%%%%%%%%%%%%%%%%%%%%%%%%%%%%%%%%%%%%%%%%%%%%%%%%%%%%%%%%%%%%%%%%%%

\subsection*{Problem}

Given a 2-D matrix, print the elements of the matrix in spiral order,
going in clockwise direction.


%%%%%%%%%%%%%%%%%%%%%%%%%%%%%%%%%%%%%%%%%%%%%%%%%%%%%%%%%%%%%%%%%%%%%%%%%%%

\subsection*{Solution}

Consider an example matrix:
%%
\begin{equation}
\label{eqn:spiral:example_matrix}
\begin{bmatrix}
0  & 1  & 2  & 3  & 4  \\[4pt]
5  & 6  & 7  & 8  & 9  \\[4pt]
10 & 11 & 12 & 13 & 14 \\[4pt]
15 & 16 & 17 & 18 & 19
\end{bmatrix}.
\end{equation}
%%
Start from the entry $\Pair{0}{0}$ in row $0$ and column $0$,
i.e.~$0$, and navigate the outer ring of the matrix.  Move along row
$0$ from left to right and collect the numbers
$\TupleE{0}{1}{2}{3}{4}$.  Start from the rightmost column at the
entry $\Pair{0}{4}$ in row $0$ and column $4$, i.e.~$4$.  Move down
from top to bottom and collect the numbers as we go along, excluding
the entry at the coordinates $\Pair{0}{4}$, and we have
$\TupleC{9}{14}{19}$.  From the bottommost row, move along from right
to left and collect the numbers $\TupleD{18}{17}{16}{15}$.  We exclude
the entry $19$ because it was collected when we traversed along the
rightmost column.  From the entry $15$ in the leftmost column, going
from bottom to top collect numbers as we go along, resulting in
$\TupleB{10}{5}$.  We exclude $15$ because it was already collected
when we traversed the bottommost row.  Furthermore, we exclude $0$
because it was collected when we moved along the topmost row.  The
outer ring of \Matrix{eqn:spiral:example_matrix} consists of the
numbers
\[
\TupleE{0}{1}{2}{3}{4},\,
\TupleC{9}{14}{19},\,
\TupleD{18}{17}{16}{15},\,
\TupleB{10}{5}.
\]
Now consider the submatrix
%%
\begin{equation}
\label{eqn:spiral:example_matrix:submatrix}
\begin{bmatrix}
6  & 7  & 8  \\[4pt]
11 & 12 & 13
\end{bmatrix}
\end{equation}
%%
of \Matrix{eqn:spiral:example_matrix}.  Use the same procedure as
above, i.e.~move along the outer ring of
\Submatrix{eqn:spiral:example_matrix:submatrix}, and we obtain the
sequence of numbers
\[
\TupleC{6}{7}{8},\,
\TupleC{13}{12}{11}.
\]
The spiral order of \Matrix{eqn:spiral:example_matrix} in clockwise
direction is the sequence of numbers
\[
\TupleE{0}{1}{2}{3}{4},\,
\TupleC{9}{14}{19},\,
\TupleD{18}{17}{16}{15},\,
\TupleB{10}{5},\,
\TupleC{6}{7}{8},\,
\TupleC{13}{12}{11}.
\]

We have $2$ exceptional cases.  The first is a (sub)matrix consisting
entirely of $1$ row, called a row matrix.  The second is a (sub)matrix
consisting entirely of $1$ column, called a column matrix.  If we have
a row matrix, then traverse the row from left to right and we are
done.  For the case of a column matrix, traverse the column from top
to bottom.  Our discussion is summarized in
\Algorithm{alg:spiral:spiral_order_clockwise}.

\begin{algorithm}[!htbp]
%%%%%%%%%%%%%%%%%%%%%%%%%%%%%%%%%%%%%%%%%%%%%%%%%%%%%%%%%%%%%%%%%%%%%%%%%%%

\section{Spiralize matrix}


%%%%%%%%%%%%%%%%%%%%%%%%%%%%%%%%%%%%%%%%%%%%%%%%%%%%%%%%%%%%%%%%%%%%%%%%%%%

\subsection*{Problem}

Given a 2-D matrix, print the elements of the matrix in spiral order,
going in clockwise direction.


%%%%%%%%%%%%%%%%%%%%%%%%%%%%%%%%%%%%%%%%%%%%%%%%%%%%%%%%%%%%%%%%%%%%%%%%%%%

\subsection*{Solution}

Consider an example matrix:
%%
\begin{equation}
\label{eqn:spiral:example_matrix}
\begin{bmatrix}
0  & 1  & 2  & 3  & 4  \\[4pt]
5  & 6  & 7  & 8  & 9  \\[4pt]
10 & 11 & 12 & 13 & 14 \\[4pt]
15 & 16 & 17 & 18 & 19
\end{bmatrix}.
\end{equation}
%%
Start from the entry $\Pair{0}{0}$ in row $0$ and column $0$,
i.e.~$0$, and navigate the outer ring of the matrix.  Move along row
$0$ from left to right and collect the numbers
$\TupleE{0}{1}{2}{3}{4}$.  Start from the rightmost column at the
entry $\Pair{0}{4}$ in row $0$ and column $4$, i.e.~$4$.  Move down
from top to bottom and collect the numbers as we go along, excluding
the entry at the coordinates $\Pair{0}{4}$, and we have
$\TupleC{9}{14}{19}$.  From the bottommost row, move along from right
to left and collect the numbers $\TupleD{18}{17}{16}{15}$.  We exclude
the entry $19$ because it was collected when we traversed along the
rightmost column.  From the entry $15$ in the leftmost column, going
from bottom to top collect numbers as we go along, resulting in
$\TupleB{10}{5}$.  We exclude $15$ because it was already collected
when we traversed the bottommost row.  Furthermore, we exclude $0$
because it was collected when we moved along the topmost row.  The
outer ring of \Matrix{eqn:spiral:example_matrix} consists of the
numbers
\[
\TupleE{0}{1}{2}{3}{4},\,
\TupleC{9}{14}{19},\,
\TupleD{18}{17}{16}{15},\,
\TupleB{10}{5}.
\]
Now consider the submatrix
%%
\begin{equation}
\label{eqn:spiral:example_matrix:submatrix}
\begin{bmatrix}
6  & 7  & 8  \\[4pt]
11 & 12 & 13
\end{bmatrix}
\end{equation}
%%
of \Matrix{eqn:spiral:example_matrix}.  Use the same procedure as
above, i.e.~move along the outer ring of
\Submatrix{eqn:spiral:example_matrix:submatrix}, and we obtain the
sequence of numbers
\[
\TupleC{6}{7}{8},\,
\TupleC{13}{12}{11}.
\]
The spiral order of \Matrix{eqn:spiral:example_matrix} in clockwise
direction is the sequence of numbers
\[
\TupleE{0}{1}{2}{3}{4},\,
\TupleC{9}{14}{19},\,
\TupleD{18}{17}{16}{15},\,
\TupleB{10}{5},\,
\TupleC{6}{7}{8},\,
\TupleC{13}{12}{11}.
\]

We have $2$ exceptional cases.  The first is a (sub)matrix consisting
entirely of $1$ row, called a row matrix.  The second is a (sub)matrix
consisting entirely of $1$ column, called a column matrix.  If we have
a row matrix, then traverse the row from left to right and we are
done.  For the case of a column matrix, traverse the column from top
to bottom.  Our discussion is summarized in
\Algorithm{alg:spiral:spiral_order_clockwise}.

\begin{algorithm}[!htbp]
\input{algorithm/spiral.tex}
\caption{%%
  The spiral order of a matrix, in clockwise direction.
}
\label{alg:spiral:spiral_order_clockwise}
\end{algorithm}

\caption{%%
  The spiral order of a matrix, in clockwise direction.
}
\label{alg:spiral:spiral_order_clockwise}
\end{algorithm}

\caption{%%
  The spiral order of a matrix, in clockwise direction.
}
\label{alg:spiral:spiral_order_clockwise}
\end{algorithm}

\caption{%%
  The spiral order of a matrix, in clockwise direction.
}
\label{alg:spiral:spiral_order_clockwise}
\end{algorithm}
