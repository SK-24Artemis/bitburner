%%%%%%%%%%%%%%%%%%%%%%%%%%%%%%%%%%%%%%%%%%%%%%%%%%%%%%%%%%%%%%%%%%%%%%%%%%%

\section{Total ways to sum}


%%%%%%%%%%%%%%%%%%%%%%%%%%%%%%%%%%%%%%%%%%%%%%%%%%%%%%%%%%%%%%%%%%%%%%%%%%%

\subsection*{Problem}

Given an integer $n > 0$, how many different ways can $n$ be written
as a sum of at least two positive integers?


%%%%%%%%%%%%%%%%%%%%%%%%%%%%%%%%%%%%%%%%%%%%%%%%%%%%%%%%%%%%%%%%%%%%%%%%%%%

\subsection*{Solution}

This is the problem of integer partition, in particular calculating
the partition number of a positive integer.\footnote{
  \url{https://en.wikipedia.org/wiki/Partition_(number_theory)}
}
The problem asks for a partition that has at least two parts, so we
must subtract $1$ from the partition number.  In number theory, the
partition number of a non-negative integer $n$ is the number of
possible partitions of $n$.  That is, the number of ways to write $n$
as a sum of positive integers.  The partition function $p(n)$ solves
the problem.\footnote{
  \url{https://en.wikipedia.org/wiki/Partition_function_(number_theory)}
}
The value of $p(n)$ can be calculated by means of a recurrence
relation due to Euler, derived from using Euler's pentagonal number
theorem.\footnote{
  \url{https://en.wikipedia.org/wiki/Pentagonal_number_theorem}
}
The recurrence relation is
%%
\begin{equation}
\label{eqn:sum:partition_function}
p(n)
=
\sum_{k \in \PP} (-1)^{k-1} \BigParen{A + B}
\end{equation}
%%
where
%%
\begin{align*}
A &= p\BigParen{n - k(3k - 1) / 2} \\[4pt]
B &= p\BigParen{n - k(3k + 1) / 2}
\end{align*}
%%
and $\PP$ is the set of all positive integers.\footnote{
  Refer to the following page for how to derive
  \Equation{eqn:sum:partition_function}:
  \url{https://www.mathpages.com/home/kmath623/kmath623.htm}
}
In practice, we only sum up to and including $k = n$.  We define the
following edge cases:
%%
\[
p(n)
=
\begin{cases}
0, & \text{if $n < 0$} \\[4pt]
1, & \text{if $n = 0$}.
\end{cases}
\]
The solution is the number $p(n) - 1$.
