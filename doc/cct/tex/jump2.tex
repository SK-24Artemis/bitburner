%%%%%%%%%%%%%%%%%%%%%%%%%%%%%%%%%%%%%%%%%%%%%%%%%%%%%%%%%%%%%%%%%%%%%%%%%%%

\section{Array jumping game II}


%%%%%%%%%%%%%%%%%%%%%%%%%%%%%%%%%%%%%%%%%%%%%%%%%%%%%%%%%%%%%%%%%%%%%%%%%%%

\subsection*{Problem}

An array has integer elements.  The integer of a cell represents the
number of positions we can jump to the right from the cell.  Starting
from the cell at index $0$, determine the minimum number of jumps
necessary to reach the last cell of the array.  Return $0$ if it is
impossible to reach the last array cell.


%%%%%%%%%%%%%%%%%%%%%%%%%%%%%%%%%%%%%%%%%%%%%%%%%%%%%%%%%%%%%%%%%%%%%%%%%%%

\subsection*{Solution}

This is similar to Array Jumping Game
from \Section{sec:array_jumping_game}.  The only difference is that
we must determine the smallest number of jumps to reach the end of the
array.  In fact, the solution presented here also solves Array Jumping
Game.  As with Array Jumping Game, we make various assumptions to
simplify the problem:

\begin{packedenumeral}
\item Each integer in the array is nonnegative.
\item We start from the zeroth index of the array.
\item Each jump must be to the right, increasing the array index.
\item If an array element is zero, we cannot jump.
\end{packedenumeral}

We interpret the array as a directed graph.  The minimum jump length
is found by computing the shortest path from the first cell to the
last cell.  Here is how to represent the array as a directed graph.
Each node of the graph is an array cell, the node ID being the
corresponding cell index.  Given two nodes $i$ and $j$, we have a
directed edge $\Pair{i}{j}$ provided we can jump from cell $i$ to cell
$j$.  If the array has $n$ cells, the maximum index value is $n - 1$.
We use Dijkstra's algorithm to find a shortest path from the source
node $s = 0$ to the target node $t = n - 1$.  The procedure is
summarized in \Algorithm{alg:jump2:array_jumping_game_II}.

\begin{algorithm}[!htbp]
%%%%%%%%%%%%%%%%%%%%%%%%%%%%%%%%%%%%%%%%%%%%%%%%%%%%%%%%%%%%%%%%%%%%%%%%%%%

\begin{algorithmic}[1]
%%
%% Input.
\Require A directed graph $G$.  A starting node $s$, where we begin
our exploration.
%%
%% Output.
\Ensure An array $\ArrayB{d}{p}$.  Here, $d$ is a map of the shortest
number of nodes in a path to a target node.  Each path starts from the
given source node $s$.  For example, $d[i]$ gives the shortest number
of nodes in a path from $s$ to node $i$.  The map $p$ gives the node
preceding a given node, in a shortest path.  For example, $p[i]$ gives
a node that directly connects to node $i$, where $p[i]$ precedes $i$.
%%
%% Algorithm body.
%%
%% Initialization.
\State $d \gets$ empty map\Comment{Or dictionary or hash table.}
\State $p \gets$ empty map
\State $d[i] \gets \infty$ for each node $i$ in $G$\Comment{$\infty$ here means a very large number.}
\State $p[i] \gets \Null$ for each node $i$ in $G$
\State $q \gets$ array of all nodes in $G$
\State $d[s] \gets 0$\Comment{Distance from $s$ to itself is $0$.}
\State $p[s] \gets \Null$\Comment{Start from $s$ so no nodes before $s$.}
\State add $s$ to end of $q$
%% Search for shortest paths from the source node to other nodes.
\While{$q$ is not empty}\Comment{Search for shortest paths to other nodes.}
  \State $u \gets$ node from $q$ such that $d[u]$ is minimal
  \State remove $u$ from $q$
  \State $N \gets $ array of all neighbours $v$ of $u$ such that $v \in q$
  \For{each $v \in N$}
    \State $t \gets d[u] + 1$
    \If{$t < d[v]$}\Comment{Found a shorter path to $v$.}
      \State $d[v] \gets t$
      \State $p[v] \gets u$
    \EndIf
  \EndFor
\EndWhile
\State \Return $\ArrayB{d}{p}$
\end{algorithmic}

\caption{%%
  Dijkstra's algorithm to find a shortest path in a directed graph.
}
\label{alg:jump2:array_jumping_game_II}
\end{algorithm}

Consider the maps $d$ and $p$ as output by
\Algorithm{alg:jump2:array_jumping_game_II}.  We have the following
cases:
%%
\begin{packedenumeral}
\item The target node $t$ is not in $d$.  There is no way to reach $t$
  from $s$.
\item Start from the target node $t$ and use $p$ to work backward to
  find a shortest path from $s$ to $t$.  If by backtracking we do not
  end up at $s$, then there are no paths from $s$ to $t$.
\item We start from $t$ and use $p$ to help us backtrack to $s$.  If
  $k$ is the number of nodes in our backtracking, including both $s$
  and $t$, the minimum number of jumps is $k - 1$.
\end{packedenumeral}
